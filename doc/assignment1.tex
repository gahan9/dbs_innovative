%%%%%%%%%%%%%%%%%%%%%%%%%%%%%%%%%%%%%%%%%
% Journal Article
% Database System
% Assignment 1: Feature Matrix
%
% Gahan M. Saraiya
% 18MCEC10
%
%%%%%%%%%%%%%%%%%%%%%%%%%%%%%%%%%%%%%%%%%
%----------------------------------------------------------------------------------------
%       PACKAGES AND OTHER DOCUMENT CONFIGURATIONS
%----------------------------------------------------------------------------------------
\documentclass[paper=letter, fontsize=12pt]{article}
\usepackage[english]{babel} % English language/hyphenation
\usepackage{amsmath,amsfonts,amsthm} % Math packages
\usepackage[utf8]{inputenc}
\usepackage{float}
\usepackage{lipsum} % Package to generate dummy text throughout this template
\usepackage{blindtext}
\usepackage{graphicx} 
\usepackage{caption}
\usepackage{subcaption}
\usepackage[sc]{mathpazo} % Use the Palatino font
\usepackage[T1]{fontenc} % Use 8-bit encoding that has 256 glyphs
\usepackage{bbding}  % to use custom itemize font
\linespread{1.05} % Line spacing - Palatino needs more space between lines
\usepackage{microtype} % Slightly tweak font spacing for aesthetics
\usepackage[hmarginratio=1:1,top=32mm,columnsep=20pt]{geometry} % Document margins
\usepackage{multicol} % Used for the two-column layout of the document
%\usepackage[hang, small,labelfont=bf,up,textfont=it,up]{caption} % Custom captions under/above floats in tables or figures
\usepackage{booktabs} % Horizontal rules in tables
\usepackage{float} % Required for tables and figures in the multi-column environment - they need to be placed in specific locations with the [H] (e.g. \begin{table}[H])
\usepackage{hyperref} % For hyperlinks in the PDF
\usepackage{lettrine} % The lettrine is the first enlarged letter at the beginning of the text
\usepackage{paralist} % Used for the compactitem environment which makes bullet points with less space between them
\usepackage{abstract} % Allows abstract customization
\renewcommand{\abstractnamefont}{\normalfont\bfseries} % Set the "Abstract" text to bold
\renewcommand{\abstracttextfont}{\normalfont\small\itshape} % Set the abstract itself to small italic text
\usepackage{titlesec} % Allows customization of titles

\renewcommand\thesection{\Roman{section}} % Roman numerals for the sections
\renewcommand\thesubsection{\Roman{subsection}} % Roman numerals for subsections

\titleformat{\section}[block]{\large\scshape\centering}{\thesection.}{1em}{} % Change the look of the section titles
\titleformat{\subsection}[block]{\large}{\thesubsection.}{1em}{} % Change the look of the section titles
\newcommand{\horrule}[1]{\rule{\linewidth}{#1}} % Create horizontal rule command with 1 argument of height
\usepackage{fancyhdr} % Headers and footers
\pagestyle{fancy} % All pages have headers and footers
\fancyhead{} % Blank out the default header
\fancyfoot{} % Blank out the default footer

\fancyhead[C]{Institute of Technology, Nirma University $\bullet$ November 2018} % Custom header text

\fancyfoot[RO,LE]{\thepage} % Custom footer text
%----------------------------------------------------------------------------------------
%       TITLE SECTION
%----------------------------------------------------------------------------------------
\title{\vspace{-15mm}\fontsize{24pt}{10pt}\selectfont\textbf{Innovative Assignment 1}} % Article title
\author{
\large
{\textsc{Gahan Saraiya (18MCEC10), Rushi Trivedi (18MCEC08), Raj Kothari (18MCEC07)}}\\[2mm]
%\thanks{A thank you or further information}\\ % Your name
\normalsize \href{mailto:18mcec10@nirmauni.ac.in}{18mcec10@nirmauni.ac.in}, % Your email address
\normalsize \href{mailto:18mcec10@nirmauni.ac.in}{18mcec08@nirmauni.ac.in}, % Your email address
\normalsize \href{mailto:18mcec10@nirmauni.ac.in}{18mcec07@nirmauni.ac.in}\\[2mm] % Your email address
}
\date{}
\hypersetup{
	colorlinks=true,
	linkcolor=blue,
	filecolor=magenta,      
	urlcolor=cyan,
	pdfauthor={Gahan Saraiya},
	pdfcreator={Gahan Saraiya},
	pdfproducer={Gahan Saraiya},
}

\usepackage{makecell}

%----------------------------------------------------------------------------------------
\begin{document}
\maketitle % Insert title
\thispagestyle{fancy} % All pages have headers and footers

\newcommand*\tick{\item[\Checkmark]}
\newcommand*\good{\CheckmarkBold}
\newcommand*\arrow{\item[$\Rightarrow$]}
\newcommand*\fail{\item[\XSolidBrush]}
\newcommand*\bad{\XSolidBrush}

\section{Introduction}
Aim of this assignment is to produce feature matrix for various multidimensional indexes.
\section{Feature Matrix}
The merits and demerits of below listed indexes are compared:
\begin{itemize}
	\item Hash Based
		\begin{itemize}
			\item Grid File
			\item Partitioned Hash
		\end{itemize}
	\item Tree Based
		\begin{itemize}
			\item Multi-key
			\item kd-Tree
			\item Quad Tree
			\item R Tree
		\end{itemize}
\end{itemize}

%\setlength{\tabcolsep}{10pt} % Default value: 6pt
\renewcommand{\arraystretch}{2} % Default value: 1
\begin{table}[!ht]
	\caption{Feature Matrix for Multidimensional Indexed}
	\newcolumntype{P}{>{\centering\arraybackslash}m{2cm}}
	\newcolumntype{R}{>{\arraybackslash}m{2cm}}
	\begin{tabular}{ R | P P | P P P P }
	
	\hline
	Query & \multicolumn{2}{c| }{\textbf{Hash Based}}& \multicolumn{4}{c}{\textbf{Tree Based}}
	\\ \cline{2-7}
	Type & Grid & Partitioned Hash & MultiKey & kd-Tree & Quad Tree & R Tree 
	\\ \hline
	Exact Match & \good & \good & \good & \good & \good & Reasonable
	\\ \hline 
	Partial Match & \good & \good & works only for first key & \good & \good & \good
	\\ \hline 
	Range & \good & \bad & \bad & \good & \good & \good
	\\ \hline 
	Nearest Neighbour & \good & \bad & \bad & Reasonable & \good & Reasonable
	\\ \hline 
	Where am I & N/A & N/A & N/A & N/A & N/A & \good 
	\\ \hline 
	\hline
	Balanced Tree & N/A & N/A & \good & \bad & \bad & \good
	\\ \hline 
	\# of empty nodes or buckets & High (if large data file) & -- & -- & -- & High [Sol: keep only Not-NULL pointer only] & N/A
	\\ \hline 
	Splitting & Easy & Hard & N/A & N/A & N/A & N/A
	\\ \hline 
	Splitting Point & Distribute Data & N/A & N/A & any point that distribute data & centre point always & N/A
	\\ \hline
	\end{tabular}
\end{table}


%----------------------------------------------------------------------------------------
%\end{multicols}
\end{document}
